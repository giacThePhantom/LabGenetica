\section*{Interazione geni-ambiente}
 \subsection{Introduzione}
 Questo esperimento valuterà la qualità della trasmissione del DNA, ma concentrandosi sui meccanismi di riparazione dello stesso in diverse condizioni ambientali, piuttosto che prendere ad esempio la sola replicazione, come nelle esperienze precedenti. Cercheremo di comprendere le eventuali esposizioni ambientali su di un gene e la bontà della replicazoine del DNA, focalizzandoci su errori di replicazioni che vengono o non vengono riparati. 
 
 \subsubsection{}
 Può succedere in vivo che una polimerasi catalizzi il nucleotide sbagliato durante il processo di replicazione, ma questa non viene considerata come una mutazione perché questo errore può essere ancora riconosciuto e risolto. I due modi in cui una cellula ripara i cosiddetti "mal appaiamenti" o "mismatch", sono, da una parte, l'attività esonucleasica intrinseca alla polimerasi, dall'altra il sistema di riparazione dei mal appaiamenti. Se si vogliono studiare mutazioni causate da mismatch, a questo punto sorge un'ambiguità. La mutazione potrebbe essere stata causata dal fallimento dell'attività esonucleasica della polimerasi, da un errore nel nucleotide incorporato erroneamente o del DNA \textit{mismatch repair} (MMR). 
 
 \subsubsection{}
 Si è costruito un sistema che permettesse di superare l'ambiguità causata da questa doppia linea di difesa contro i mal appaiamenti. Sì è visto che in sequeunze omonucleotidiche la polimerasi tende ad introdurre un nucleotide in più, anche a causa di una potenziale diversa topologia assunta dai due filamenti di DNA in presenza di lunghe sequenze ripetute. Se la sequenza da copiare omonucleotidica, soprattutto A e T diventa lunga, maggiore di 7, la correzione di bozze sembra essere totalmente inefficace, la polimerasi non si rende conto di absi extraelica che dà una distorsione dell'elica, già uscita dal sito catalitico. Rimane un solo meccanismo di riparazione, MMR appunto, ed è su questa osservazoine che si basa il nostro esperimento. 
 
 \subsection{•}
 Non tutte le sequenze di DNA sono replicabili con la stessa efficacia, quelle che potrebbero rappresentare siti di mutazione vengono chiamate \textit{at risk motifs}.  Vogliamo testare, sfruttando queste sequenze più fragili se alcune condizioni nel terreno, come agenti mutageni, possono interferire con il sistema MMR. Per farlo in vivo abbiamo bisogno di un saggio reporter, nel nostro caso un gnee importante nella biosintesi della lisina. Avremo a disposizione un ceppo il quale è stato sottoposto ad un intervento di ingegnerizzazione, non fatto da noi am ereditato, nelq uale ilg gene lys2 è stato inettivato da uno stretch omonucleotidico di A (precisamente 14). Le A effetttivamente inserite sono comunque 13. Scivolamento del modulo di lettura bisogna interpretarlo come uno shift di 13 A. Il quadro di lettura del nostro ceppo sarà quindi sfalsato a valle. dato che è posizionato abbastanza al centro della sequenza codificante, il nostro ceppo non sarà più ing rado di crescere in un terreno senza lisina. Noi partiremo da un reporter che ha bisgono di lisina per crescere su piastra. Ma se ci fosse un erorre di sintesi del DNA, nella quale la polimerasi lascia indietro una adenina, ecco che avremmo una discendenza nella quale le cellule hanno 12 adenine esogene, restaurando il quadro di lettura. Il mutante tornerà a crescere in un terreno senza lisina. Vederemo apparire effetto della reversione. La quantificazoine della reversione ci consente di stimare il numero di malappaiamenti risotli grazie a MMR.
 
 \subsection{•}
 La preparazione all'esperimento si è svolto nell'arco dei tre giorni. 
 
 
 