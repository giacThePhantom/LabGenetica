\section*{Organismo Modello \emph{Saccharomyces cerevisiae}}

	\subsection*{Panoramica}
	Il microorganismo \emph{Saccharomyces cerevisiae}, comunemente conosciuto come lievito del pane \`e un organismo eucariote unicellulare.
	\`E stato ampiamente utilizzato per studiare la genetica degli eucarioti, ma il suo utilizzo risale all'antichit\`a.
	Veniva infatti usato per la produzione di pane, birra e, in tempi pi\`u moderni, di bicarbonati.
	Nel \date{1857} Pasteur lo identifica come il microorganismo responsabile della fermentazione, mentre il suo utilizzo nell'analisi genetica inizia nel \date{1935}.
	Le numerose ricerche compiute su esso hanno permesso di caratterizzare accuratamente i suoi geni.
	Allo stesso tempo, la natura unicellulare del lievito lo rende adatto alle tecniche molecolari sviluppate per i batteri.
	Questo organismo permette pertanto di combinare la genetica classica e la biologia molecolare, rendendolo un modello potente per lo studio dei sistemi genetici eucarioti

	\subsection*{Vantaggi come organismo genetico modello}
	Il lievito, oltre ad essere un organismo eucariote con sistemi genetici simili a quelli di altri organismi pi\`u complessi come l'uomo, \`e anche unicellulare, cosa che lo rende semplice da maneggiare come i batteri.
	Ha tempi di coltura veloci in laboratorio: una divisione cellulare avviene in $90$ minuti, in condizioni normali di crescita.
	Essendo comunque un organismo eucariote possiede comunque un nucleo distinto con cromosomi lineari multipli assemblati in cromatina e il citoplasma \`e dotato dell'intero spettro di organelli intracellulari e di strutture citoscheletriche.
	Pu\`o esistere sia in forma aploide che diploide.
	Quando si trova in forma aploide le cellule possiedono un solo allele in ogni locus.
	Questo allele verr\`a pertanto espresso, rendendo impossibile un mascheramento dell'espressione di altri alleli da parte di alleli dominanti.
	Questo permette una facile identificazione degli alleli recessivi nelle cellule aploidi.
	In un secondo momento si pu\`o complementare il lievito rendendolo diploide in modo da studiare l'interazione di questi alleli con altri.
	Un'altra caratteristica del lievito \`e che al termine della meiosi tutti i gameti prodotti si trovano in un \emph{asco} e rimangono separati dai gameti prodotti nelle altre divisioni meiotiche.
	Le quattro cellule contenute in un asco sono dette \emph{tetradi}.
	L'analisi genetica delle tetradi in \emph{S. cerevisiae} consente di osservare direttamente gli effetti delle singole divisioni meiotiche e di identificare pi\`u facilmente gli eventi di crossing-over.
	Numerosi analisi genetiche hanno identificato migliaia di mutanti e molte potenti tecniche molecolari sviluppate per manipolare le sequenze genetiche nei batteri sono state adattate per essere usate nel lievito.
	Infine le cellule del lievito possiedono molti geni presenti anche nell'uomo e in altri eucarioti complessi con funzionalit\`a identiche o simili.
	Si nota pertanto come lo studio genetico delle cellule di lievito spesso contribuisce alla comprensione di meccanismi di organismi pi\`u complessi, uomo compreso.

	\subsection*{Il ciclo vitale del lievito}
	\emph{Saccharomyces cerevisiae} pu\`o pertanto esistere sotto forma di cellule aploidi o diploidi.

		\subsubsection*{Forma aploide}
		La forma aploide compare tipicamente in condizioni di carenza di nutrienti.
		Il lievito si riproduce per via mitotica, producendo due cellule aploidi identiche al genitore per gemmazione.

		\subsubsection*{Forma diploide}
		La forma diploide nasce dopo riproduzione sessuata del lievito.
		Si distinguono due tipi sessuali: \emph{a} e \emph{$\alpha$}.
		Due cellule appartenenti a tipi sessuali diversi si uniscono e fondono i nuclei, dando origine a una cellula diploide, in grado di generare cellule diploidi geneticamente identiche per gemmazione.
		La carenza di sostanze nutritive induce le cellule a subire meiosi, formando quattro nuclei aploidi in cellule diverse e infine $4$ spore aploidi.

		\subsubsection*{Cambiamenti di forma}
		Durante la progressione del ciclo cellulare le cellule di \emph{Saccharomyces cerevisiae} cambiano forma.
		Immediatamente dopo il rilascio dalla cellula madre, la figlia ha una forma leggermente ellittica.
		Durante il ciclo cellulare sviluppa una piccola gemma da cui si origina una nuova cellula.
		La spora cresce fino a raggiungere le stesse dimensioni della cellula madre da cui origine, viene rilasciata e ricomincia il ciclo.
		Essendo l'insorgenza di una nuova gemma strettamente legata all'inizio della replicazione del DNA, osservazioni al microscopio sono in grado di fornire molte informazioni riguardo gli eventi interni alla cellula.
		
	\subsection*{Genoma del lievito}
		\emph{Saccharomyces cerevisiae} contiene $16$ paia di cromosomi eucariotici.
		Il genoma contiene $12$ milioni di paia di basi oltre a $2$-$3$ milioni di paia di geni di \emph{rRNA}.
		\`E stato il primo organismo eucariote il cui genoma \`e stato completamente sequenziato.
		In esso sono stati identificati circa \num{6000} geni.
		La sua analisi genomica \`e potenziata dal fatto che quando un DNA lineare con estremit\`a omologhe viene introdotto nel lievito si osserva un incremento notevole della frequenza di ricombinazione omologa.
		Questa propriet\`a pu\`o essere usata per causare precise mutazioni nel genoma.
		Il $25\%$ dei geni sono in comune con l'uomo.
		La dimensione media dei geni \`e di $1.5Kbp$, con una media di $0.03$ introni per gene.
		Possiede inoltri pochi trasposoni.

		\subsubsection*{Plasmidi del lievito}

			\paragraph*{Plasmide \emph{$\mathbf{2\mu}$}}
			Le cellule del lievito in natura possiedono un plasmide circolare \emph{2$\mu$}.
			\`E lungo \num{6300} paia di basi e viene trasmesso durante mitosi e meiosi alle cellule figlie.
			Possiede un origine di replicazione riconosciuta dal sistema di replicazione del lievito ed \`e in grado di replicarsi in modo autonomo nella cellula.
			Modifiche al plasmide lo hanno reso un vettore efficiente per il trasferimento di geni nel lievito.

			\paragraph*{Plasmidi batterici}
			Plasmidi batterici modificati possono essere usati come vettori per il trasferimento dei geni.
			Alcuni di questi sono in grado di dare ricombinazione omologa con il cromosoma del lievito trasferendo le loro sequenze al genoma di \emph{Saccharomyces cerevisiae}.

	\subsection*{Ambiti di utilizzo}
	\emph{Saccharomyces cerevisiae} viene utilizzato in vari ambiti:
	\begin{multicols}{2}
		\begin{itemize}
			\item Studio dell'invecchiamento.
			\item Studio dell'apoptosi.
			\item Controllo del ciclo cellulare.
			\item Ricombinazione.
			\item Studi sull'interazione tra proteine.
			\item Genetica dei tipi sessuali: studio della risposta agli ormoni riproduttivi del tipo sessuale opposto.
		\end{itemize}
	\end{multicols}
