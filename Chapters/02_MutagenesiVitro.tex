\section*{Mutagenesi \emph{in vitro}}

	\subsection*{Indroduzione}

        Nella prima esperienza si vuole verificare l'efficacia e la precisione della polimerasi (ci hanno detto il nome della polimerasi in particolare?)
        in presenza di sostanze tossiche e/o inibitorie.
        L'organismo modello utilizzato è stato il lievito Saccharomyces Cerevisiae in forma aploide, uno dei principali Organismi modello, è unicellulare, ha un breve ciclo
        di vita e può essere indotto a una forma diploide. 
        Nel lievito esistono due tipi sessuali indicati con mat-a e mat-alpha, e se messi in condizioni favorevoli, possono fondersi in cellule diploidi.

        Si vuole studiare la replicazione del DNA in vitro, per testare la capacità di una polimerasi di rimanere efficace anche in presenza di sostanze inibitorie o tossiche.
        Si userà la PCR aggiungendo un ingrediente che di solito non viene utilizzato, il cloruro di manganese per rendere il processo più difficile.

        In vitro viene utilizzato il calore per separare le due eliche di DNA mentre in natura questo compito è affidato a un enzima elicasi. 
        Si abbassa poi la temperatura per far legare il primer a una sequenza precisa per permettere alla PCR di trovare un 3'-OH da cui iniziare la replicazione.
        Questo processo verrà ripetuto più volte.

        Gli ioni magnesio sono usati come catalizzatori della polimerizzazione favorendo la sostituzione nucleofilica del 3'-OH libero del primer con il fosfato del nucleotide.
        
        Il manganese può competere con il magnesio per l'accesso al sito allosterico? Se si, può cambiarne l'efficienza?

        Proveremo a rispondere aggiungendo alla reazione di PCR cloruro di manganese.

        La sequenza di DNA da amplificare inoltre non sarà scelta a caso, ma sarà la sequenza che condifica per un gene reporter.

        Nella cellula verrà inserito anche un plasmide linearizzato, pRDI22, che dovrà avere un'origine di replicazione riconoscibile dall'organismo ospitante e una regione
        centromerica, in modo da poter essere scambiato per un plasmide del lievito.
        Il plasmide conterrà il gene reporter, per conferirgli la capacità di essere replicato all'interno della cellula, e il gene leu, per cui il ceppo di lievito utilizzato
        è auxotrofo, questo permetterà di selezionarlo.

        Oltre all'auxotrofia per la leucina, che servirà a verificare l'efficacia di pRDI22, il ceppo utilizzato sarà auxotrofo per l'adenina, in questo modo, si può verificare
        il funzionamento di P53.

	\subsection*{Risultati attesi}

        1) Il manganese può influenzare, in positivo o negativo la PCR?
        2) Può il manganese introdurre delle mutazioni?

            Per darci una risposta dobbiamo inserire il risultato della PCR in cellule vive, per verificare la funzionalità del gene aplificato.
            Se il gene reporter è replicato in maniera corretta, i lieviti con questo gene cresceranno di colore bianco, se il DNA viene mutato, il lievito crescerà rosa/rosso.

   	\subsection*{Risultati}
        Sono state usate quattro diverse concentrazioni di manganese cloruro, 0M, 0,25M, 0,5M e 1M, più una provetta di controllo negativo senza DNA per verificare la
        purezza dei reagenti.
	\begin{table}[H]
		\centering
		\begin{tabular}{|c|c|c|c|c|}
			\hline
			\makecell{Concentrazione di \emph{$MnCl_2$}} & $0M$ & $0.25M$ & $0.5M$ & $1M$\\
			\hline
			\makecell{Conta totale} & $2672CFU$ & $1822CFU$ & $1778CFU$ & $599CFU$ \\
			\hline
			\makecell{Colonie mutate} & $236CFU$ & $1044CFU$ & $1672CFU$ & $5564CFU$\\
			\hline
			\makecell{Tasso di mutazione} & $8.83\%$ & $57.3\%$ & $94\%$ & $94.16\%$\\
			\hline
		\end{tabular}
		\caption{Conta delle colonie}
		\label{tab}
	\end{table}


	\subsection*{Considerazioni finali}
	I risultati ottenuti, come si vede nella tabella~\ref{tab}, dimostrano come, non solo il cloruro di manganese riesca a competere con il mangesio per l'accesso al sito allosterico,
        ma anche quanto riesca a influire sia sull'efficienza della replicazione, che sulla sua precisione.
        Al crescere della concentrazione di manganese cloruro infatti, la conta totale di cellule in piastra, diminuisce, mentre aumenta la percentuale delle cellule mutate.
        (La presenza di un certo tasso di mutazione anche nella soluzione senza manganese è imputabile all'affidabilità della PCR stessa)

