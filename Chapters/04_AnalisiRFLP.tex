\section*{Analisi \emph{RFLP} di polimorfismi a singolo nucleotide}

	\subsection*{Introduzione}
	Questa esperineza serve per andare a valutare la variabilità genetica dovuta a singoli polimorfismi di singoli nucleotidi (SNPs). Per SNPs si intende una variazione genetica e rappresenta un cambio di nucleotide all'interno della nostra sequenza. In linea di massima gli SNPs possono essere visti come delle mutazioni dato che sono comunque una variazione nella sequenza, ma a cambiare è la presenza con cui si presentano nella popolazione. Le mutazioni, infatti, sono molto più rare  rispetto agli SNPs che si trovano in almeno nell'1$\%$ della popolazione. Inoltre, gli SNPs non sono sempre associati a delle patologie, infatti, nella maggior parte dei casi vanno a determinare alcune nostre caratteristiche. 
	Durante questa esperienza si è andati a valutare la capacità di una persona di percepire più o meno il gusto dell'amaro tramite lo studio del recettore TAS2R38. 
	
	    \subsubsection*{Recettore TAS2R38}
	    Questo recettore è in grado di andare a percepire i glucosinalati che sono dei composti presenti in alcuni alimenti che mangiamo, come i broccoli. 
	    TAS2R38 si trova sul cromosoma 7 e al suo interno presenta cinque differenti polimorfismi, ma i tre più comuni si trovano in determinate posizioni della proteina all'aminoacido 49, 263 e 296. Questi tre polimorfismi danno origine a due fenotipi, ma a tre aplotipi: 
	    	\begin{itemize}
				\item Taster con aplotipo PAV;
				\item Taster con aplotipo AVV sono meno frequenti (3$\%$), ma hanno la stessa capacità di percepire il gusto di quelli che vengono considerati taster più frequenti;
				\item Non-taster con aplotipo AVI. 
		\end{itemize}
	
	    
	    \subsubsection*{Procedimento}
	    Per verificare la presenza di SNPs è stata aplificata una porzione del gene TAS2R38 mediante una PCR. Per andare a discriminare i diversi genotipi si è deciso di utilizzare un enzima di restrizione (Fun4H1). 
	    Si sono caricate le PCR digerite con enzimi di restrizione e sotto ciascuna di queste è stata caricata la PCR non digerita. Per poter visualizzare meglio i risultati è stato aggiunto anche un colorante blu. 
	   
	    \subsubsection*{Risultati aspettati}
    
        Dalla PCR non digerita con enzima di restrizione non ci si aspetta di avere alcuna indicazione su variazioni a livello nucleotidico. 
        Nelle PCR digerite con enzimi di restrizione, invece, ci si aspetta di vedere alcune differnze. Se tutti i cinque siti di restrizione sono integri ci si aspetta un determinato tipo di bandeggio, mentre se in alcuni casi l'enzima di restrizione non riuscisse a riconoscere il sito di restrizione quello che si nota è un bandeggio completamente differente. 
        Ci si aspetta che si presentino tutte e tre le condizioni, dato che i campioni che abbiamo provengono da tre individui con diverso aplotipo. Quindi quello che dovremmo riuscire ad apprezzare dopo la corsa du gel di agarosio è che:
        \begin{itemize}
            \item Il taster presenterà le bande più basse; 
            \item Il non taster presenterà le bande più alte;
            \item Il taster meno frequente presenterà entrambe le componenti. 
        \end{itemize}
	
	\subsection*{Risultati}
		\begin{figure}[H]
			\centering
			\includegraphics[width=\textwidth]{./Pics/RFLP/RFLP\ gruppoD\ bio1(SYBR Safe).jpg}
			\caption{Elettroforesi su gel d'agarosio dei risultati della PCR}
			\label{fig10}
		\end{figure}

	\subsection*{Considerazioni finali}
	
	Facendo riferimento agli ultimi due gruppi si possono fare alcune considerazioni finali su la riuscita o meno dell'esperienza. 
	Come previsto, dalla PCR non digerita non si ha avuto alcuna indicazione sulle variazioni a livello nucleotidico.
	Per quanto riguarda il primo dei due gruppi si può dire che l'esperienza è avvenuta con successo, in quanto si sono ottenuti i risultati attesi. Unica nota va fatta sulla corsa del DNA dell'individuo taster in quanto era presente meno colorante. 
	Per quanto riguarda il secondo gruppo non ci sono stati problemi per l'individuo taster e l'individuo taster meno frequente, ma purtroppo non si è potuta apprezzare la corsa del non taster. 
	Per entrambi si può comunque concludere che l'aplotipo AVV ha la componente di entrambi e per questo motivo è considerato eterozigote. 
